% (c) 2002 Matthew Boedicker <mboedick@mboedick.org> (original author) http://mboedick.org
% (c) 2003-2007 David J. Grant <davidgrant-at-gmail.com> http://www.davidgrant.ca
% (c) 2008 Nathaniel Johnston <nathaniel@nathanieljohnston.com> http://www.nathanieljohnston.com
% (c) 2011 Scott Clark <sc932@cornell.edu> http://cam.cornell.edu/~sc932
%

%This work is licensed under the Creative Commons Attribution-Noncommercial-Share Alike 2.5 License. To view a copy of this license, visit http://creativecommons.org/licenses/by-nc-sa/2.5/ or send a letter to Creative Commons, 543 Howard Street, 5th Floor, San Francisco, California, 94105, USA.

\documentclass[letterpaper,11pt]{article}
\newlength{\outerbordwidth}
\pagestyle{empty}
\raggedbottom
\raggedright
\usepackage[svgnames]{xcolor}
\usepackage{framed}
\usepackage{tocloft}
\usepackage{etoolbox}
\robustify\cftdotfill

%-----------------------------------------------------------
%Edit these values as you see fit
\setlength{\outerbordwidth}{3pt}  % Width of border outside of title bars
\definecolor{shadecolor}{gray}{0.75}  % Outer background color of title bars (0 = black, 1 = white)
\definecolor{shadecolorB}{gray}{0.93}  % Inner background color of title bars

%-----------------------------------------------------------
%Margin setup
\setlength{\evensidemargin}{-0.25in}
\setlength{\headheight}{-0.25in}
\setlength{\headsep}{0in}
\setlength{\oddsidemargin}{-0.25in}
\setlength{\paperheight}{11in}
\setlength{\paperwidth}{8.5in}
\setlength{\tabcolsep}{0in}
\setlength{\textheight}{9.75in}
\setlength{\textwidth}{7in}
\setlength{\topmargin}{-0.3in}
\setlength{\topskip}{0in}
\setlength{\voffset}{0.1in}

%-----------------------------------------------------------
%Custom commands
\newcommand{\resitem}[1]{\item #1 \vspace{-2pt}}
\newcommand{\resheading}[1]{\vspace{8pt}
  \parbox{\textwidth}{\setlength{\FrameSep}{\outerbordwidth}
    \begin{shaded}

\setlength{\fboxsep}{0pt}\framebox[\textwidth][l]{\setlength{\fboxsep}{4pt}\fcolorbox{shadecolorB}{shadecolorB}{\textbf{\sffamily{\mbox{~}\makebox[6.762in][l]{\large #1} \vphantom{p\^{E}}}}}}
    \end{shaded}
  }\vspace{-5pt}
}

\newcommand{\ressubheading}[4]{
\begin{tabular*}{6.5in}{l@{\cftdotfill{\cftsecdotsep}\extracolsep{\fill}}r}
		\textbf{#1} & #2 \\
		\textit{#3} & \textit{#4} \\
\end{tabular*}\vspace{-6pt}}

%-----------------------------------------------------------
\begin{document}

\begin{tabular*}{7in}{l@{\extracolsep{\fill}}r}

\textbf{{\Large Kurian C Kurian}} & \textbf{\today} \\
\texttt{kurianck@yahoo.in} \\
\texttt{9446055131}

\end{tabular*}


%%%%%%%%%%%%%%%%%%%%%%%%%%%%%%

\resheading{Career Overview}

%%%%%%%%%%%%%%%%%%%%%%%%%%%%%%

\begin{itemize}
\item
    \ressubheading{HiFx IT private limited}{}{}{}
    \begin{itemize}
    \resitem{}
    \end{itemize}
    
	\ressubheading{Bharat Petroleum Corporation Limited}{Mumbai,India}{}{2009 - 2011}
	\begin{itemize}
	\resitem{\textbf{Study of Power Consumption in Chain Conveyors:} In a \emph{never stopping} production line chain conveyors play a significant role. They take the product through various
	stages of processing. However due the very nature of being constantly on, they also represent
	one of the constant demands of power. This can be draining if the chain conveyors are not optimally
	designed. At Bhitoni LPG bottling plant I carried out a study, to find if any of the motors were
	undergoing any major stress. We used the measurement of current drawn and the recommended value for
	each motor to ascertain whether there was an excess stress. This was preceded by minor adjustments in conveyor layout and subsequently better performance from the conveyor system.}
	\resitem{\textbf{Installation of Remote Operating Valves(R.O.V):} R.O.V's are pneumatic valves that
	run on air pressure of 8 kg/cm. They are used to remotely open or close a valve which its 
	advisable to do so from a distance. In Bhitoni LPG bottling plant, I was member of a team of officers
	and workers who installed ROV's on tanker unloading bay's. Thus ensuring that they can be operated 
	from a safe distance if need be. Since the operation would cost essential working hours of plant, it 
	needed meticulous planning and had to be conducted without interludes spanning three days and 2 
	nights.}
	\end{itemize}
\end{itemize}


%%%%%%%%%%%%%%%%%%%%%%%%%%%%%%

\resheading{Research Interest}

%%%%%%%%%%%%%%%%%%%%%%%%%%%%%%

\begin{itemize}

\item
	\ressubheading{Use Of Charactersitic Sets In 2-D Descrete Systems}{IIT-Guwahati}{}{2012-2014}
	\begin{itemize}
		\resitem{\textbf{Controllability in the 1-D:}Controllaiblity and observability are well defined for 1-D systems. It is taught in basic courses of electrical engineering everywhere. An essential part of design of any elctrical,mechanical or electromechincal sytem is to ensure that they are controllable and are stable for a known range of operating conditions}
		\resitem{\textbf{Controllablility for 2-D Discrete Systems:}The essential question when acertianing controllability of systems that are goverened by two independent parameters is the question of natural ordering. While future,past,present are obvious in 1-D systems what defines such things for the 2-D system? without such defenitions can we define controllability or observability? what can one say about a system that gives bounded output along one of the independent axes but is unbounded on the other? Such questions were taken up in the paper of Dr Jan C Willems \cite{73561}}.This paper deals with systems as a set of trajectories,which is departure from the conventional treatment of system as  black box. In subsequent times efforts were made to truly implement the idea of behavioral systems and to find universal defition to their controllability and observability. The paper by of Maria E Vlacher\cite{valcher},Zampieri \cite{zampieri},HK Pillai and D. Pal \cite{PP} are examples of the on going research in this feild. 

		two dimensional systems are an explicit example where behavioral \cite{73561}approach works and yeilds meaningful deductions for controllability and observablilty. Study of charachteristic sets is cruciual in this context because they form the part of
		defining observability and controllability for behaviorals. My thesis gives an interpretation of the conclusions in \cite{PP} which leads to a surprising result that charachteristic sets can indeed be more compactly defined than previously thought. 
	\end{itemize}
\end{itemize}

\resheading{Writing and Awards}

\begin{itemize}

\item {\bf 2016 Forbes 30 Under 30:} Enterprise Tech. \texttt{http://onforb.es/1OILpBZ}


\end{itemize}

%%%%%%%%%%%%%%%%%%%%%%%%%%%%%%

\resheading{Skills}

%%%%%%%%%%%%%%%%%%%%%%%%%%%%%%

\begin{itemize}
\item {\bf Control Systems:} Designing feedback systems for linear control systems. Writing code to represent nonlinear systems in MATLAB.Finding solutions to nonlinear equations using MATLAB. 
\item {\bf Tech Stack:} Python, C++,javascript,Go,php
\item{\bf Frameworks:} Django,API design in Go
\end{itemize}

%%%%%%%%%%%%%%%%%%%%%%%%%%%%%%

\resheading{Selected Open Source Projects}

%%%%%%%%%%%%%%%%%%%%%%%%%%%%%%

\begin{itemize}

\item
    \ressubheading{SigOpt Examples (\texttt{github.com/sigopt/sigopt-examples})}{Python}{Examples of using SigOpt to tune ML algorithms.}{2014 - current}
    \begin{itemize}
        \resitem{Examples of using SigOpt to tune everything from sklearn to beating Vegas and beyond.}
	\end{itemize}

\item
    \ressubheading{MOE: Metric Optimization Engine (\texttt{github.com/Yelp/MOE})}{Python, C++, CUDA}{A global, black box optimization engine for real world metric optimization}{2010 - 2015}
    \begin{itemize}
        \resitem{Implemented throughout Yelp, optimizing ad metrics. 2nd most popular open source project.}
        \resitem{Talk: \texttt{bit.ly/1plYZA2}, Slides: \texttt{slidesha.re/1zOrOJy}, Blog: \texttt{bit.ly/1x73xdr}}
        \resitem{Presented to executives, universities, conferences and companies around the country.}
	\end{itemize}

\item
    \ressubheading{ALE: Assembly Likelihood Estimator (\texttt{github.com/sc932/ALE})}{C, Python}{Probabilistic evaluation of genome assemblies}{2010 - 2013}
    \begin{itemize}
        \resitem{Uses statistical function to score and rank genome assemblies, published in Bioinformatics}
	\end{itemize}

\end{itemize}
\bibliography{cv}
\bibliographystyle{alpha}
\end{document}
